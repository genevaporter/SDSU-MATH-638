\documentclass[12pt]{article}
\usepackage[utf8x]{inputenc}
\usepackage{amsmath}
\usepackage{graphicx}
\usepackage{indentfirst}

\title{Project Proposal: \\ Turing Instability and Pattern Formation}
\author{Jose Ramirez-Huitron and Geneva Porter}
\date{27 February 2019}

\begin{document}
\maketitle

\vspace{3cm}

\section*{Intro}

Reaction-diffusion systems can be simple, but they can also yield solutions describing complicated pattern formations. For this project, we will be using Turing analysis techniques to investigate pattern formation via a reaction-diffusion system. Specifically, we will be examining the Gierer-Meinhardt model\cite{Gierer1972}, a PDE system describing activator-inhibitor relationships. 

The structure of our project paper will include both analytical and numerical solution conditions. First, we will examine the form of basic reaction-diffusion systems and the additional parameters of the Gierer-Meinhardt model. Next, we will do a basic stability analysis by finding the eigenvalues of the model with and without the diffusion terms, then evaluating under which conditions patterning occurs. Applying non-dimensionalization analysis will then allow us to perform a simplified bifurcation study. After examining the model analytically, we will use a finite difference scheme to approximate a numerical solution at varying parameter values. Pattern formation visualizations will follow with these results.

\pagebreak

\section*{Model Analysis}

The classic reaction-diffusion system has the form:

$$
\frac{\partial u}{\partial t}=D_u\frac{\partial^2u}{\partial x^2}+f(u,v)
$$

$$
\frac{\partial v}{\partial t}=D_v\frac{\partial^2v}{\partial x^2}+g(u,v)
$$

\vspace{3mm}

With $D_u$ and $D_v$ being the diffusion constants and $f$ and $g$ being the reaction functions. Here $u$ and $v$ are functions of position and time.  Patterns emerge when we have a {\it Turing instability}, also called a {\it diffusion-driven instability}\cite{ShapeOfMath}. A Turing instability occurs when a zero-diffusion system in a stable steady state switches to an unstable steady state when a diffusion term is included. Our goal is to find the conditions for which a Turing instability occurs. 

The Gierer-Meinhardt model elaborates on the classic system. This model is given by:

\begin{align*}
    \frac{\partial u}{\partial t} & = D_u\frac{\partial^2u}{\partial t^2} + \rho\frac{u^2}{v} - a_1u + a_2 \\
    \frac{\partial v}{\partial t} & = D_v\frac{\partial^2v}{\partial t^2} + \rho u^2 - b_1v + b_2 \\
\end{align*}

As we can see, there is a quadratic added to the diffusion term for each equation. Here the function $u$ designated the behavior of the activator substance, while $v$ is the inhibitor substance. The quadratic equations describe the inter-dependent relationship between the production and decay of two substances. The production rate is given by $\rho$, with decay rates $a_1$ and $b_1$ and concentration constants $a_2$ and $b_2$. More about these parameters and their respective roles will be explained in the final project.

After finding patterning conditions of this model analytically, we will visualize solutions to this system numerically.

\pagebreak

\section*{Numerical Analysis}

In order to reduce issues with consistency and stability, an implicit, second-order finite difference scheme shall be use to solve the Gierer-Meinhardt system numerically. These scheme properties, along with a well-posedness assumption, should guarantee convergence by the Lax-Richtmyer Equivalence Theorem. \\

Lax-Richtmyer Equivalence Theorem:
\textit{A consistent finite difference scheme for a partial differential equation for which the initial value problem is well-posed is convergent if and only if it is stable.} \cite{Strikwerda2004} \\

In general, the scheme needs to work on multidimensional systems of parabolic partial differential equations. Therefore, the Crank-Nicolson scheme, a readily known finite difference scheme, meets the criteria:


\begin{align*}
  \frac{v^{n+1}_m-v^{n-1}_m}{k}=\frac{1}{2} & b\frac{v^{n+1}_{m+1}-2v^{n+1}_{m}+v^{n+1}_{m-1}}{h^2} \\ +&\frac{1}{2}  b\frac{v^{n}_{m+1}-2v^{n}_{m}+v^{n}_{m-1}}{h^2} \\  +& \frac{f^{n+1}_m+f^{n-1_m}}{2}  
\end{align*}


Here $V^n_m$ refers to the $n^{th}$ time step of the $m^{th}$ spatial point in the system.The variables $h$, $k$, and $b$ are the space step, time step, and diffusivity, respectively.\cite{Strikwerda2004}

The patterns expected to form should be dependent on the size of the domain; as such this is the parameter that should be varied to determine the influence of the size of the domain on the pattern formation.

 \section*{Discussion}\\
This section of the project will cover how the numerical results reflect what was uncovered in the Turing Analysis of the Gierer-Meinhardt Model.

\pagebreak

\nocite{*}
\bibliographystyle{acm}
\bibliography{638bib}

\end{document}